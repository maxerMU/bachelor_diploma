
%\begin{essay}{}

 
%\end{essay}

\setcounter{page}{3}

\chapter*{РЕФЕРАТ}
Расчетно-пояснительная записка к курсовой работе содержит  \begin{NoHyper}\pageref{LastPage}\end{NoHyper} страниц, \totfig~иллюстраций, \tottab~таблиц, 12 источников, 1 приложение.

Научно-исследовательская работа представляет собой изучение предметной области классификации объектов на аэрофотоснимках, описание основных методов, а также преимуществ и недостатков каждого из них. Рассмотрены различные подходы решения поставленной задачи. Представлено описание алгоритмов детерминированного подхода, дерева решений и нейронных сетей. Подробно разобраны преимущества и недостатки перцептрона, сверточных нейронных сетей и капсульных нейронных сетей.

Ключевые слова: нейрон, перцептрон, сверточная нейронная сеть, капсульная нейронная сеть.