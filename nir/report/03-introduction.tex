\chapter*{ВВЕДЕНИЕ}
\addcontentsline{toc}{chapter}{ВВЕДЕНИЕ}


Одной из основных задач, возникающих при обработке изображений, является распознавание и классификация различных объектов на снимке. В роли объектов может быть различная техника: вертолеты, самолеты, машины, корабли.

Обработка в реальном времени полезна для задач контроля движения судов, поиска объектов на местности в случае аварии и крушения, предотвращение таких ситуаций, картографии.

Целью данной работы является анализ существующих методов распознавания и классификации объектов с фотоснимков.

Для достижения поставленной цели требуется выполнить следующие задачи:
\begin{itemize}
	\item определить термины, связанные с предметной областью распознавания объектов;
	\item провести классификацию и разобрать методы распознавания с фотоснимков;
	\item определить недостатки и преимущества каждого из них.
\end{itemize}