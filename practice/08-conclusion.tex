\chapter*{ЗАКЛЮЧЕНИЕ}
\addcontentsline{toc}{chapter}{ЗАКЛЮЧЕНИЕ}

В рамках производственной практики было разработано программное обеспечение для распознавания летательных аппаратов с использованием нейронных сетей.

Были выбраны и реализованы средства распознавания летательных аппаратов. Для этого были выбраны методы аугментации обучающих данных, направленные на увеличение количества и разнообразия данных для обучения модели. 

Были созданы модели для детектирования и классификации летательных аппаратов и реализован алгоритм их обучения. После этого был проведен ряд тестов, чтобы оценить качество модели. Итоговая точность модели классификации на тестовой выборке составила 84 процента. Точность детектирования на тестовой выборке составила 90 процентов.

Кроме того, был создан графический интерфейс, который позволяет пользователям использовать обученные модели для распознавания типов летательных аппаратов. После этого было проведено ручное тестирование обученных моделей, чтобы убедиться в правильности их работы.