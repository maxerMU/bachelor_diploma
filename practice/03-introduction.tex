\chapter*{ВВЕДЕНИЕ}
\addcontentsline{toc}{chapter}{ВВЕДЕНИЕ}

Одной из основных задач, возникающих при обработке изображений, является распознавание различных объектов на снимке. В роли объектов может быть различная техника: вертолеты, самолеты, машины, корабли.

Обработка в реальном времени полезна для задач контроля движения судов, поиска объектов на местности в случае аварии и крушения, предотвращение таких ситуаций, картографии.

Распознавание объектов с аэрофотоснимков полезно во время непрерывно ведущегося наблюдения с воздуха. В этом случае можно вести наблюдение с воздуха и в автоматическом режиме выдавать информацию о происходящем на земле.

К примеру, в настоящее время один из способов разведки территорий -- съемка с беспилотных летательных аппаратов. Человек в ручном режиме не всегда может обработать информацию, которая непрерывным потоком поступает с беспилотника, ведущего разведку, поэтому весь поток информации нужно обрабатывать автоматически. 

В случае разведки летательной техники важным является определение моделей самолетов, стоящий в аэропорту. Такая классификации в случае военной техники поможет сделать вывод о готовящейся тактике и силах стороны, территория которой разведывается.

Во время выполнения выпускной квалификационной работы был разработан метод распознавания летательных аппаратов с аэрофотоснимков с использованием нейронных сетей.