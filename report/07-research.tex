\chapter{Исследовательский раздел}

\section{Сравнение методов оптимизации}

Одной из проблем стохастического градиентного спуска является неизменяемая во время обучения скорость обучения. Постоянная скорость обучения может привести к следующим проблемам: если ее значение будет выбрано слишком низким, то модель будет дольше сходиться и требовать большего числа итераций для достижения оптимального решения, если ее значение будет слишком большим, то модель может расходиться и на каждой итерации проходить мимо глобального минимума.

Для решения этой проблемы нужно использовать адаптивно настраивающуюся скорость обучения. Значение скорости обучения для каждого параметра должно настраиваться адаптивно, исходя из правила, что чем больше значение ошибки, тем больше должна быть скорость обучения. Увеличение скорости обучения при больших значениях ошибки дает возможность перескочить через локальные минимумы, а ее уменьшение при малых значениях не дает модели на каждой итерации перескакивать через глобальный минимум.

Для адаптивного изменения весов модели можно использовать алгоритм RMSProps. Этот алгоритм работает по следующим правилам:
\begin{itemize}
	\item на каждой итерации для каждого параметра считается экспоненциальное скользящее среднее градиента с учетом всей истории обучения;
	\item при помощи полученных значений для каждого параметра вычисляется скорость обучения и производится обновление весов модели.
\end{itemize}

Экспоненциальное скользящее среднее на очередной итерации высчитывается по формуле \ref{formula:rmsprops_ema}
\begin{equation}\label{formula:rmsprops_ema}
E_{\rm{t}} = \beta * g_{\rm{t}}^2 + (1 - \beta) * E_{\rm{t-1}},
\end{equation}

где $E_{\rm{t}}$ -- новое полученное значение экспоненциального скользящего среднего, $E_{\rm{t-1}}$ -- значение, полученное на предыдущей итерации, $\beta$ -- настраиваемый коэффициент, $g_{\rm{t}}$ -- градиент функции потерь по соответствующему параметру.

После этого веса обновляются с использованием соотношения \ref{formula:rmsprops_w}
\begin{equation}\label{formula:rmsprops_w}
w_{\rm{t}} = w_{\rm{t-1}} - \frac{\eta}{\sqrt{E_{\rm{t}}}}g,
\end{equation}
где $w_{\rm{t}}$ -- новое полученное значение параметра модели, $w_{\rm{t-1}}$ -- предыдущее значение этого параметра, $\eta$ -- скорость обучения, $E_{\rm{t}}$ -- значение экспоненциального скользящего среднего для этого параметра.

На рисунке \ref{img:sgd_rmsprops_test} представлены графики сравнения значения функции потерь моделей, обучаемых с помощью стохастического градиентного спуска и RMSProps, в зависимости от номера эпохи обучения.

%\begin{figure}[H]
%	\begin{center}
%		\begin{tikzpicture}
%		\begin{axis}[
%		xlabel = {номер эпохи},
%		ylabel = {точность},
%		legend pos=south east,
%		ymajorgrids=true,
%		width=12cm
%		]   
%		\addplot[color=blue, mark=square] table[x index=0, y index=1] {inc/data/sgd_train.dat};
		
%		\addplot[color=red, mark=*] table[x index=0, y index=1] {inc/data/rmsprop_train.dat};
		
		
%		\addlegendentry{SGD}
%		\addlegendentry{RMSProps}
		
%		\end{axis}
%		\end{tikzpicture}
%		\caption{Сравнение моделей, обученных на RMPSProp и SGD, на тренировочной выборке}
%		\label{img:sgd_rmsprops_training}
%	\end{center}
%\end{figure}
\begin{figure}[H]
	\begin{center}
		\begin{tikzpicture}
		\begin{axis}[
		xlabel = {номер эпохи},
		ylabel = {значение функции потерь},
		legend pos=south west,
		ymajorgrids=true,
		width=12cm
		]   
		\addplot[color=blue, mark=square] table[x index=0, y index=1] {inc/data/loss_sgd.dat};
		
		\addplot[color=red, mark=*] table[x index=0, y index=1] {inc/data/loss_rms.dat};
		
		
		\addlegendentry{SGD}
		\addlegendentry{RMSProps}
		
		\end{axis}
		\end{tikzpicture}
		\caption{Сравнение функций потерь моделей, обученных на RMPSProp и SGD}
		\label{img:sgd_rmsprops_test}
	\end{center}
\end{figure}

Из полученных результатов можно сделать вывод о том, что модель, которая обучалась с использованием градиентного спуска попала в локальный минимум функции потерь и перестала обучаться. 

Модель, которая обучалась с помощью RMSProps за счет адаптивного вычисления скорости обучения на каждой итерации не попала в этот локальный минимум и продолжила обучение.

Итоговая точность моделей, обучаемых с помощью градиентного спуска и RMSProp, на тестовой выборке составила 21 процент и 39 процентов соответственно.

Для улучшения сходимости модели при адаптивном обновлении скорости обучения можно считать экспоненциальное скользящее среднее не только по квадрату градиента, но и по самому значению и использовать оба полученных значения при подсчете новой скорости обучения на каждой итерации. Такой подход реализован в алгоритме Adam.

Первый и второй моменты высчитываются по формулам \ref{formula:adamema1} и \ref{formula:adamema2} соответственно

\begin{equation}\label{formula:adamema1}
m_{\rm{t}} = \beta_{\rm{1}} * g_{\rm{t}} + (1 - \beta_{\rm{1}}) * m_{\rm{t-1}},
\end{equation}

\begin{equation}\label{formula:adamema2}
v_{\rm{t}} = \beta_{\rm{2}} * g_{\rm{t}}^2 + (1 - \beta_{\rm{2}}) * v_{\rm{t-1}},
\end{equation}

где $m_{\rm{t}}$ и $v_{\rm{t}}$ -- первый и второй моменты в соответствующий момент времени, $\beta_{\rm{1}}$ и $\beta_{\rm{2}}$ -- настраиваемые коэффициента, $g$ -- градиент функции потерь по соответствующему параметру.

Для увеличения влияния истинных значений градиента на начальных этапах к моментам применяется корректировка по формулам \ref{formula:adamcorr1} и \ref{formula:adamcorr2}.

\begin{equation}\label{formula:adamcorr1}
\hat{m_{\rm{t}}} = \frac{m_{\rm{t}}}{1 - \beta_{\rm{1}}^t},
\end{equation}

\begin{equation}\label{formula:adamcorr2}
\hat{v_{\rm{t}}} = \frac{v_{\rm{t}}}{1 - \beta_{\rm{2}}^t},
\end{equation}

Итоговое обновление весов осуществляется по формуле 
\ref{formula:adam_w}
\begin{equation}\label{formula:adam_w}
w_{\rm{t}} = w_{\rm{t-1}} - \eta\frac{\hat{m_{\rm{t}}}}{\sqrt{\hat{v_{\rm{t}}}} + \epsilon},
\end{equation}
где $w_{\rm{t}}$ -- новое полученной значение параметра модели, $w_{\rm{t-1}}$ -- предыдущее значение этого параметра, $\eta$ -- скорость обучения, $\epsilon$ -- поправка, защищающая от деления на ноль.

На рисунках \ref{img:adam_rmsprops_test} представлены графики функции потерь моделей, обученных с помощью Adam и RMSProps, в зависимости от номера эпохи обучения.

%TODO check graphics
%\begin{figure}[H]
%	\begin{center}
%		\begin{tikzpicture}
%		\begin{axis}[
%		xlabel = {номер эпохи},
%		ylabel = {точность},
%		legend pos=south east,
%		ymajorgrids=true,
%		width=12cm
%		]   
%		\addplot[color=blue, mark=square] table[x index=0, y index=1] {inc/data/adam_train.dat};
%		
%		\addplot[color=red, mark=*] table[x index=0, y index=1] {inc/data/rmsprop_train.dat};
%		
%		
%		\addlegendentry{ADAM}
%		\addlegendentry{RMSProps}
%		
%		\end{axis}
%		\end{tikzpicture}
%		\caption{Сравнение моделей, обученных на RMPSProp и ADAM, на тренировочной выборке}
%		\label{img:adam_rmsprops_training}
%	\end{center}
%\end{figure}
%\begin{figure}[H]
%	\begin{center}
%		\begin{tikzpicture}
%		\begin{axis}[
%		xlabel = {номер эпохи},
%		ylabel = {точность},
%		legend pos=south east,
%		ymajorgrids=true,
%		width=12cm
%		]   
%		\addplot[color=blue, mark=square] table[x index=0, y index=1] {inc/data/adam_test.dat};
%		
%		\addplot[color=red, mark=*] table[x index=0, y index=1] {inc/data/rmsprop_test.dat};
%		
%		
%		\addlegendentry{ADAM}
%		\addlegendentry{RMSProps}
%		
%		\end{axis}
%		\end{tikzpicture}
%		\caption{Сравнение моделей, обученных на RMPSProp и ADAM, на тестовой выборке}
%		\label{img:adam_rmsprops_test}
%	\end{center}
%\end{figure}
\begin{figure}[H]
	\begin{center}
		\begin{tikzpicture}
		\begin{axis}[
		xlabel = {номер эпохи},
		ylabel = {значение функции потерь},
		legend pos=south west,
		ymajorgrids=true,
		width=12cm
		]   
		\addplot[color=blue, mark=square] table[x index=0, y index=1] {inc/data/loss_adam.dat};
		
		\addplot[color=red, mark=*] table[x index=0, y index=1] {inc/data/loss_rms.dat};
		
		
		\addlegendentry{Adam}
		\addlegendentry{RMSProps}
		
		\end{axis}
		\end{tikzpicture}
		\caption{Сравнение функций потерь моделей, обученных на RMPSProp и Adam}
		\label{img:adam_rmsprops_test}
	\end{center}
\end{figure}

Из графиков видно, что функция потерь, которая оптимизировалась с помощью алгоритма Adam сходится быстрее и не попадает в локальные минимумы. Итоговая точность модели, которая обучалась с помощью алгоритма Adam, составила 84 процента на тестовой выборке.

\section{Модификация модели классификации}

После каждого из слоев пуллинга в архитектуре модели классификации был добавлен слой дропаут, который с некоторой заданной вероятностью $p$ исключает нейрон из обучения на одну итерацию. Такой прием приводит к тому, что модель обучается на разных комбинациях активных нейронов, что уменьшает вероятность переобучения модели.

Графики зависимости точности модели на тестовой выборке от номера эпохи обучения с использованием dropout и без приведены на рисунке \ref{img:dropout_test}.

\begin{figure}[H]
	\begin{center}
		\begin{tikzpicture}
		\begin{axis}[
		xlabel = {номер эпохи},
		ylabel = {точность},
		legend pos=south east,
		ymajorgrids=true,
		width=12cm
		]   
		\addplot[color=blue, mark=square] table[x index=0, y index=1] {inc/data/model_test.dat};
		
		\addplot[color=red, mark=*] table[x index=0, y index=1] {inc/data/adam_test_no_drop.dat};
		
		
		\addlegendentry{с dropout}
		\addlegendentry{без dropout}
		
		\end{axis}
		\end{tikzpicture}
		\caption{Сравнение моделей, обученных с использованием dropout и без, на тестовой выборке}
		\label{img:dropout_test}
	\end{center}
\end{figure}

Из графиков видно, что после 6 эпохи обучения, модель которая обучалась без использования dropout стала обучаться на шум на тренировочной выборке, в связи с чем точность модели на тестовой выборке начала падать.

Помимо добавления слоев дропаута из архитектуры сети был убран один из сверточных слоев. Точность модели на тестовой выборке после этого осталась прежней, а число обучаемых параметров было уменьшено на 600 тысяч. Уменьшение число обучаемых параметров увеличивает скорость обучения сети, снижает ограничения на вычислительные ресурсы, а также уменьшает вероятность переобучения модели.

\section{Вывод}
В архитектуру нейронной сети были внесены изменения, которые увеличивают скорость обучения модели, снижают ограничения на вычислительные ресурсы, а также решают проблему переобучения модели.

При сравнении различных оптимизаторов было выяснено, что скорость обучения сети должна адаптивно настраиваться в зависимости от текущего значения ошибки по правилу: чем больше ошибка, тем больше скорость обучения. В ином случае модель может дольше сходиться при малом значении скорости обучения, либо расходиться при больших значениях этого параметра.

Алгоритмы Adam и RMSProps используются для адаптивной настройки скорости обучения. В RMSprops учитывается экспоненциальное скользящее среднее по квадрату градиента для учета истории обучения при обновлении весов модели. В алгоритме Adam также учитывается экспоненциальное скользящее среднее и по первому моменту, что позволяет ему достичь большей сходимости при обучении модели.

Таким образом, разработанный метод показывает наибольшую точность при обучении моделей с помощью алгоритма Adam.
