\chapter*{ВВЕДЕНИЕ}
\addcontentsline{toc}{chapter}{ВВЕДЕНИЕ}

Одной из основных задач, возникающих при обработке изображений, является классификация различных объектов на снимке. В роли объектов может быть различная техника: вертолеты, самолеты, машины, корабли.

Обработка в реальном времени полезна для задач контроля движения судов, поиска объектов на местности в случае аварии и крушения, предотвращение таких ситуаций, картографии.

Классификация объектов с аэрофотоснимков полезна во время непрерывно ведущегося наблюдения с воздуха. В этом случае можно вести наблюдение с воздуха и в автоматическом режиме выдавать информацию о происходящем на земле.

Целью данной работы является разработка метода распознавания летательный аппаратов с аэрофотоснимков.

Для достижения поставленной цели требуется выполнить следующие задачи:
\begin{itemize}
	\item определить термины, связанные с предметной областью;
	\item провести анализ методов классификации летальной техники на аэрофотоснимках;
	\item определить критерии сравнения методов;
	\item провести их сравнительный анализ;
	\item разработать метод классификации летальной техники на аэрофотоснимках;
	\item спроектировать программное обеспечение для реализации разработанного метода;
	\item реализовать спроектированный метод;
	\item провести исследование точности распознавания модели на тестовой выборке при различных подходах к обучению.
	
\end{itemize}